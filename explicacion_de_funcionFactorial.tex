\documentclass{article}

% Language setting
% Replace `english' with e.g. `spanish' to change the document language
\usepackage[spanish]{babel}

% Set page size and margins
% Replace `letterpaper' with `a4paper' for UK/EU standard size
\usepackage[letterpaper,top=2cm,bottom=2cm,left=3cm,right=3cm,marginparwidth=1.75cm]{geometry}

% Useful packages
\usepackage{amsmath}
\usepackage{graphicx}
\usepackage[colorlinks=true, allcolors=blue]{hyperref}

\title{Funcion Factorial.py}
\author{Mateo Rojas}

\begin{document}
\maketitle

\begin{abstract}
Este es un programa que devuelve el factorial de un número ingresado. Creado por Mateo Rojas.
\end{abstract}

\section{Palabras del autor}

Hay que ser honestos, este programa no sirve para mucho, puedes sacar el factorial de cuantos números quieras, podrías usar la función en algún otro programa que lo requiera, pero toda la funcionalidad y más ya existe en el módulo math. 

\section{¿Como funciona la función?}
Primero se recibe un número.

Segundo, se comprueba si el número es menor que cero (0) si lo es, automáticamente se retorna cero(0).

En el caso de números mayores o iguales que cero(0) se declara una variable \textit{total} y se le asigna el valor uno(1) después se inicia un bucle while, en este, el valor de \textit{total} se multiplica por el número recibido, entonces se le resta uno(1) al número recibido, cuando el número recibido sea igual a uno(1) el bucle termina y se retorna el valor de \textit{total} el cual es el valor de x!



\end{document}